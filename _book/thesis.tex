\PassOptionsToPackage{unicode=true}{hyperref} % options for packages loaded elsewhere
\PassOptionsToPackage{hyphens}{url}
%
\documentclass[12pt,openright,oneside,a4paper,chapter=TITLE,section=TITLE,subsection=Title,english,french,spanish,portugues,sumario=tradicional]{04-class-files/abntex2}
\usepackage{lmodern}
\usepackage{amssymb,amsmath}
\usepackage{ifxetex,ifluatex}
\usepackage{fixltx2e} % provides \textsubscript
\ifnum 0\ifxetex 1\fi\ifluatex 1\fi=0 % if pdftex
  \usepackage[T1]{fontenc}
  \usepackage[utf8]{inputenc}
  \usepackage{textcomp} % provides euro and other symbols
\else % if luatex or xelatex
  \usepackage{unicode-math}
  \defaultfontfeatures{Ligatures=TeX,Scale=MatchLowercase}
\fi
% use upquote if available, for straight quotes in verbatim environments
\IfFileExists{upquote.sty}{\usepackage{upquote}}{}
% use microtype if available
\IfFileExists{microtype.sty}{%
\usepackage[]{microtype}
\UseMicrotypeSet[protrusion]{basicmath} % disable protrusion for tt fonts
}{}
\IfFileExists{parskip.sty}{%
\usepackage{parskip}
}{% else
\setlength{\parindent}{0pt}
\setlength{\parskip}{6pt plus 2pt minus 1pt}
}
\usepackage{hyperref}
\hypersetup{
            pdfauthor={Nome do Discente},
            pdfborder={0 0 0},
            breaklinks=true}
\urlstyle{same}  % don't use monospace font for urls
\usepackage{color}
\usepackage{fancyvrb}
\newcommand{\VerbBar}{|}
\newcommand{\VERB}{\Verb[commandchars=\\\{\}]}
\DefineVerbatimEnvironment{Highlighting}{Verbatim}{commandchars=\\\{\}}
% Add ',fontsize=\small' for more characters per line
\usepackage{framed}
\definecolor{shadecolor}{RGB}{248,248,248}
\newenvironment{Shaded}{\begin{snugshade}}{\end{snugshade}}
\newcommand{\AlertTok}[1]{\textcolor[rgb]{0.94,0.16,0.16}{#1}}
\newcommand{\AnnotationTok}[1]{\textcolor[rgb]{0.56,0.35,0.01}{\textbf{\textit{#1}}}}
\newcommand{\AttributeTok}[1]{\textcolor[rgb]{0.77,0.63,0.00}{#1}}
\newcommand{\BaseNTok}[1]{\textcolor[rgb]{0.00,0.00,0.81}{#1}}
\newcommand{\BuiltInTok}[1]{#1}
\newcommand{\CharTok}[1]{\textcolor[rgb]{0.31,0.60,0.02}{#1}}
\newcommand{\CommentTok}[1]{\textcolor[rgb]{0.56,0.35,0.01}{\textit{#1}}}
\newcommand{\CommentVarTok}[1]{\textcolor[rgb]{0.56,0.35,0.01}{\textbf{\textit{#1}}}}
\newcommand{\ConstantTok}[1]{\textcolor[rgb]{0.00,0.00,0.00}{#1}}
\newcommand{\ControlFlowTok}[1]{\textcolor[rgb]{0.13,0.29,0.53}{\textbf{#1}}}
\newcommand{\DataTypeTok}[1]{\textcolor[rgb]{0.13,0.29,0.53}{#1}}
\newcommand{\DecValTok}[1]{\textcolor[rgb]{0.00,0.00,0.81}{#1}}
\newcommand{\DocumentationTok}[1]{\textcolor[rgb]{0.56,0.35,0.01}{\textbf{\textit{#1}}}}
\newcommand{\ErrorTok}[1]{\textcolor[rgb]{0.64,0.00,0.00}{\textbf{#1}}}
\newcommand{\ExtensionTok}[1]{#1}
\newcommand{\FloatTok}[1]{\textcolor[rgb]{0.00,0.00,0.81}{#1}}
\newcommand{\FunctionTok}[1]{\textcolor[rgb]{0.00,0.00,0.00}{#1}}
\newcommand{\ImportTok}[1]{#1}
\newcommand{\InformationTok}[1]{\textcolor[rgb]{0.56,0.35,0.01}{\textbf{\textit{#1}}}}
\newcommand{\KeywordTok}[1]{\textcolor[rgb]{0.13,0.29,0.53}{\textbf{#1}}}
\newcommand{\NormalTok}[1]{#1}
\newcommand{\OperatorTok}[1]{\textcolor[rgb]{0.81,0.36,0.00}{\textbf{#1}}}
\newcommand{\OtherTok}[1]{\textcolor[rgb]{0.56,0.35,0.01}{#1}}
\newcommand{\PreprocessorTok}[1]{\textcolor[rgb]{0.56,0.35,0.01}{\textit{#1}}}
\newcommand{\RegionMarkerTok}[1]{#1}
\newcommand{\SpecialCharTok}[1]{\textcolor[rgb]{0.00,0.00,0.00}{#1}}
\newcommand{\SpecialStringTok}[1]{\textcolor[rgb]{0.31,0.60,0.02}{#1}}
\newcommand{\StringTok}[1]{\textcolor[rgb]{0.31,0.60,0.02}{#1}}
\newcommand{\VariableTok}[1]{\textcolor[rgb]{0.00,0.00,0.00}{#1}}
\newcommand{\VerbatimStringTok}[1]{\textcolor[rgb]{0.31,0.60,0.02}{#1}}
\newcommand{\WarningTok}[1]{\textcolor[rgb]{0.56,0.35,0.01}{\textbf{\textit{#1}}}}
\usepackage{longtable,booktabs}
% Fix footnotes in tables (requires footnote package)
\IfFileExists{footnote.sty}{\usepackage{footnote}\makesavenoteenv{longtable}}{}
\usepackage{graphicx,grffile}
\makeatletter
\def\maxwidth{\ifdim\Gin@nat@width>\linewidth\linewidth\else\Gin@nat@width\fi}
\def\maxheight{\ifdim\Gin@nat@height>\textheight\textheight\else\Gin@nat@height\fi}
\makeatother
% Scale images if necessary, so that they will not overflow the page
% margins by default, and it is still possible to overwrite the defaults
% using explicit options in \includegraphics[width, height, ...]{}
\setkeys{Gin}{width=\maxwidth,height=\maxheight,keepaspectratio}
\setlength{\emergencystretch}{3em}  % prevent overfull lines
\providecommand{\tightlist}{%
  \setlength{\itemsep}{0pt}\setlength{\parskip}{0pt}}
\setcounter{secnumdepth}{5}
% Redefines (sub)paragraphs to behave more like sections
\ifx\paragraph\undefined\else
\let\oldparagraph\paragraph
\renewcommand{\paragraph}[1]{\oldparagraph{#1}\mbox{}}
\fi
\ifx\subparagraph\undefined\else
\let\oldsubparagraph\subparagraph
\renewcommand{\subparagraph}[1]{\oldsubparagraph{#1}\mbox{}}
\fi

% set default figure placement to htbp
\makeatletter
\def\fps@figure{htbp}
\makeatother

\input{00-input-datas/00-dados}
\input{09-packages/00-pacotes}
\makeindex
\usepackage{helvet}
\renewcommand{\familydefault}{\sfdefault}
\DeclareUnicodeCharacter{0301}{*}
\usepackage[style=abnt,]{biblatex}
\addbibresource{10-references/referencias.bib}

\author{Nome do Discente}
\date{2020}

\begin{document}

\ifthenelse{\equal{\terCapa}{Sim}}{
\imprimircapa}{}

\ifthenelse{\equal{\terFolhaRosto}{Sim}}{
\imprimirfolhaderosto*}{}

\ifthenelse{\equal{\terFichaCatalografica}{Sim}}
 {\insereFichaCatalografica{}\cleardoublepage}
 {}

\ifthenelse{\equal{\terErrata}{Sim}}
 {\begin{errata}%\color{blue}
   \imprimirerrata
  \end{errata}}
 {}

\ifthenelse{\equal{\terTermoAprovacao}{Sim}}{
\insereAprovacao}{}

\ifthenelse{\equal{\terDedicatoria}{Sim}}{
\begin{dedicatoria}
   \vspace*{\fill}
   \centering
   \noindent
   \DedicatoriaTexto
   \vspace*{\fill}
\end{dedicatoria}
}{}

\ifthenelse{\equal{\terAgradecimentos}{Sim}}
 {\begin{agradecimentos}
    \AgradecimentosTexto
  \end{agradecimentos}
  }{}

\ifthenelse{\equal{\terEpigrafe}{Sim}}{
\begin{epigrafe}
    \vspace*{\fill}
	\begin{flushright}
        \EpigrafeTexto
	\end{flushright}
\end{epigrafe}
}{}

\ifthenelse{\equal{\terResumos}{Sim}}{
\begin{resumo}
    \ResumoTexto
    

   \noindent 
   \textbf{Palavras-chaves}: \PalavraschaveTexto
\end{resumo}

\begin{resumo}[ABSTRACT]
 \begin{otherlanguage*}{english}
   \AbstractTexto
   
   \noindent 
   \textbf{Key-words}: \KeywordsTexto
 \end{otherlanguage*}
\end{resumo}



\ifthenelse{\equal{\Resume}{}}
{}
{
 \begin{resumo}[RESUME]%Résumé
  \begin{otherlanguage*}{french}
     \Resume
     
   \noindent      
    \textbf{Mots clés}: \Motscles
  \end{otherlanguage*}
 \end{resumo}
} 


\ifthenelse{\equal{\Resume}{}}{}
{ \begin{resumo}[RESUMEN]
  \begin{otherlanguage*}{spanish}
    \Resumen 
   
   \noindent    
    \textbf{Palabras clave}: \Palabrasclave
  \end{otherlanguage*}
 \end{resumo}
}
}{}

\ifthenelse{\equal{\terListaFiguras}{Sim}}{
\pdfbookmark[0]{\listfigurename}{lof}
\listoffigures*
\cleardoublepage
}{}

\ifthenelse{\equal{\terListaTabelas}{Sim}}{
\listoftables*
\cleardoublepage
}{}

\ifthenelse{\equal{\terSiglasAbrev}{Sim}}{
    \imprimirlistadesiglas
    \cleardoublepage
    \imprimirlistadesimbolos
    \cleardoublepage
 }{}

\ifthenelse{\equal{\terSumario}{Sim}}{
\tableofcontents*
}{}

\textual

\pagestyle{simple}

\chapter[INTRODUÇÃO]{INTRODUÇÃO}

Este projeto é uma adaptação para o ambiente R Markdown utilizando a ferramenta bookdown combinando o modelo canônico de trabalho acadêmicos da \abnTeX e a adapatação para UFPR realizada por Emilio E Kawamura.

Nesse sentido esse documento de exemplo conterá as redações idendicas destes dois trabalhos, com acréscimo de exemplos de utilização e visualização de códigos em linguagem R.

Este documento e seu código-fonte são exemplos de referência de uso da classe
\textcite{abntex2} e do pacote \textcite{abntex2cite}. O documento
exemplifica a elaboração de trabalho acadêmico (tese, dissertação e outros do
gênero) produzido conforme a ABNT NBR 14724:2011 \emph{Informação e documentação
- Trabalhos acadêmicos - Apresentação}.

A expressão ``Modelo Canônico'' é utilizada para indicar que \abnTeX~não é
modelo específico de nenhuma universidade ou instituição, mas que implementa tão
somente os requisitos das normas da ABNT. Uma lista completa das normas
observadas pelo \abnTeX~é apresentada em \textcite{abntex2classe}.

Sinta-se convidado a participar do projeto \abnTeX! Acesse o site do projeto em
\url{http://www.abntex.net.br/}. Também fique livre para conhecer,
estudar, alterar e redistribuir o trabalho do \abnTeX, desde que os arquivos
modificados tenham seus nomes alterados e que os créditos sejam dados aos
autores originais, nos termos da ``The \LaTeX~Project Public
License''\footnote{\url{http://www.latex-project.org/lppl.txt}}.

Encorajamos que sejam realizadas customizações específicas deste exemplo para
universidades e outras instituições --- como capas, folha de aprovação, etc.
Porém, recomendamos que ao invés de se alterar diretamente os arquivos do
\abnTeX, distribua-se arquivos com as respectivas customizações.
Isso permite que futuras versões do \abnTeX\textasciitilde{}não se tornem automaticamente
incompatíveis com as customizações promovidas.

Este documento deve ser utilizado como complemento dos manuais do \abnTeX~
\cite{abntex2classe,abntex2cite,abntex2cite-alf} e da classe \textcite{memoir}
\cite{memoir}.

Esperamos, sinceramente, que o \abnTeX~aprimore a qualidade do trabalho que
você produzirá, de modo que o principal esforço seja concentrado no principal:
na contribuição científica.

Equipe \abnTeX 

Lauro César Araujo

Para obter os melhores resultados, compile este modelo usando a seguinte sequência de passos:

\begin{quote}
\begin{footnotesize}
\begin{verbatim}
pdflatex  main          // compilação inicial
bibtex main             // processa referências bibliográficas
pdflatex  main          // compilação final
\end{verbatim}
\end{footnotesize}
\end{quote}

ou

\begin{quote}
\begin{footnotesize}
\begin{verbatim}
make                    // faz tudo...
\end{verbatim}
\end{footnotesize}
\end{quote}

Os principais itens considerados na formatação deste documento foram:

\begin{itemize}

\item Papel em formato A4, com margens de 20 mm à direita e embaixo, 30 mm nos demais lados. Não devem ser usados cabeçalhos ou rodapés além dos que estão aqui propostos.

\item O texto principal do documento escrito em 12 pontos. O fonte principal do texto pode ser selecionado no arquivo \verb#packages.tex#.

\item Código-fonte, listagens e textos similares são formatados em fonte Courier 12 ou 10 pontos.

\item O espaçamento padrão entre linhas é 1,5 linhas (1 linha na versão final). Não inserir espaços adicionais entre parágrafos normais. Figuras, tabelas, listagens e listas de itens devem ter um espaço adicional antes e após os mesmos.

\item As páginas iniciais não são numeradas.

\item O corpo do texto é numerado com algarismos arábicos (1, 2, 3, ...) a partir da introdução, ate o final do documento. Os números de página devem estar situados no alto à direita (páginas direitas) ou à esquerda (páginas esquerdas).

\item Expressões em inglês, grego, latim ou outras línguas devem ser enfatizadas em itálico, como \emph{sui generis} ou \emph{scheduling} (use o comando \verb#\emph{...}#).

\item Para reforçar algo, deve-se usar somente \textbf{negrito}. \underline{Sublinhado} ou MAIÚSCULAS não devem ser usados como forma de ênfase!

\item As notas de rodapé também têm um modelo\footnote{As notas de rodapé dever ser escritas em tamanho 10 pt, numeradas em arábico.}. Notas de rodapé servem para fazer algum comentário paralelo; não as use para colocar URLs, referências bibliográficas ou significado de siglas.

\end{itemize}

Felizmente o \LaTeX~resolve a maior parte dessas questões!

\part{Referenciais teóricos}

\chapter{Lorem ipsum dolor sit amet}

\section{Aliquam vestibulum fringilla lorem}

\lipsum[1]

\lipsum[2-3]

\part{Preparação da pesquisa}

\include{03-tex-files/abntex2-modelo-include-comandos}

\chapter{Conteúdos específicos do modelo de trabalho acadêmico}\label{cap_trabalho_academico}

\section{Quadros}

Este modelo vem com o ambiente \texttt{quadro} e impressão de Lista de quadros
configurados por padrão. Verifique um exemplo de utilização:

\begin{quadro}[htb]
\caption{\label{quadro_exemplo}Exemplo de quadro}
\begin{tabular}{|c|c|c|c|}
	\hline
	\textbf{Pessoa} & \textbf{Idade} & \textbf{Peso} & \textbf{Altura} \\ \hline
	Marcos & 26    & 68   & 178    \\ \hline
	Ivone  & 22    & 57   & 162    \\ \hline
	...    & ...   & ...  & ...    \\ \hline
	Sueli  & 40    & 65   & 153    \\ \hline
\end{tabular}
\fonte{Autor.}
\end{quadro}

Este parágrafo apresenta como referenciar o quadro no texto, requisito
obrigatório da ABNT.
Primeira opção, utilizando \texttt{autoref}: Ver o \autoref{quadro_exemplo}.
Segunda opção, utilizando \texttt{ref}: Ver o Quadro \ref{quadro_exemplo}.

\part{Resultados}

\chapter{Dados e Análises com Códigos R}

Abaixo é um teste para geração de Tabelas com código R dentro do environment Latex

\begin{tabular}{c|c}
\hline
Simbolo & Descrição\\
\hline
A & Descrição A\\
\hline
B & Descrição B\\
\hline
C & Descrição C\\
\hline
\end{tabular}

Calculando a média no R:

\begin{Shaded}
\begin{Highlighting}[]
\NormalTok{vector <-}\StringTok{ }\KeywordTok{c}\NormalTok{(}\DecValTok{2}\NormalTok{,}\DecValTok{4}\NormalTok{,}\DecValTok{6}\NormalTok{,}\DecValTok{8}\NormalTok{,}\DecValTok{10}\NormalTok{)}
\NormalTok{mean.vector <-}\StringTok{ }\KeywordTok{mean}\NormalTok{(vector)}
\KeywordTok{cat}\NormalTok{(}\StringTok{'A média do vetor é:'}\NormalTok{, mean.vector)}
\end{Highlighting}
\end{Shaded}

A média do vetor é: 6

\begin{Shaded}
\begin{Highlighting}[]
\KeywordTok{summary}\NormalTok{(}\KeywordTok{lm}\NormalTok{(}\DataTypeTok{formula =}\NormalTok{ mpg }\OperatorTok{~}\StringTok{ }\NormalTok{., }\DataTypeTok{data =}\NormalTok{ mtcars))}\OperatorTok{$}\NormalTok{coefficients}
\end{Highlighting}
\end{Shaded}

\begin{verbatim}
##                Estimate  Std. Error    t value   Pr(>|t|)
## (Intercept) 12.30337416 18.71788443  0.6573058 0.51812440
## cyl         -0.11144048  1.04502336 -0.1066392 0.91608738
## disp         0.01333524  0.01785750  0.7467585 0.46348865
## hp          -0.02148212  0.02176858 -0.9868407 0.33495531
## drat         0.78711097  1.63537307  0.4813036 0.63527790
## wt          -3.71530393  1.89441430 -1.9611887 0.06325215
## qsec         0.82104075  0.73084480  1.1234133 0.27394127
## vs           0.31776281  2.10450861  0.1509915 0.88142347
## am           2.52022689  2.05665055  1.2254035 0.23398971
## gear         0.65541302  1.49325996  0.4389142 0.66520643
## carb        -0.19941925  0.82875250 -0.2406258 0.81217871
\end{verbatim}

\begin{Shaded}
\begin{Highlighting}[]
\KeywordTok{str}\NormalTok{(mtcars)}
\end{Highlighting}
\end{Shaded}

Modelo de gráfico em código R:

\begin{center}\includegraphics{thesis_files/figure-latex/unnamed-chunk-6-1} \end{center}
\begin{figure}[!htb]
\centering
\caption{Gráfico GGPlot para Teste}
\label{fig:ggplot}
\end{figure}

\section{Regressão com Código R}

Seguem abaixo exemplos - reproduções de partes do curso de Regressão Linear da Universidade Johns Hopkins - de inserção de códigos R para captura, tratamento, exploração, análises econométrica, predições, construção de algorítimos e formulas.

\hypertarget{swiss-fertility-data}{%
\subsection{Swiss fertility data}\label{swiss-fertility-data}}

\begin{Shaded}
\begin{Highlighting}[]
\KeywordTok{library}\NormalTok{(datasets); }\KeywordTok{data}\NormalTok{(swiss); }\KeywordTok{require}\NormalTok{(stats); }\KeywordTok{require}\NormalTok{(graphics)}
\KeywordTok{pairs}\NormalTok{(swiss, }\DataTypeTok{panel =}\NormalTok{ panel.smooth, }\DataTypeTok{main =} \StringTok{"Swiss data"}\NormalTok{, }
      \DataTypeTok{col =} \DecValTok{3} \OperatorTok{+}\StringTok{ }\NormalTok{(swiss}\OperatorTok{$}\NormalTok{Catholic }\OperatorTok{>}\StringTok{ }\DecValTok{50}\NormalTok{))}
\end{Highlighting}
\end{Shaded}

\begin{center}\includegraphics{thesis_files/figure-latex/unnamed-chunk-7-1} \end{center}

\hypertarget{swiss}{%
\subsection{\texorpdfstring{\texttt{?swiss}}{?swiss}}\label{swiss}}

\hypertarget{description}{%
\subsubsection{Description}\label{description}}

Standardized fertility measure and socio-economic indicators for each of 47 French-speaking provinces of Switzerland at about 1888.

A data frame with 47 observations on 6 variables, each of which is in percent, i.e., in {[}0, 100{]}.

\begin{itemize}
\tightlist
\item
  {[},1{]} Fertility Ig, ` common standardized fertility measure'
\item
  {[},2{]} Agriculture \% of males involved in agriculture as occupation
\item
  {[},3{]} Examination \% draftees receiving highest mark on army examination
\item
  {[},4{]} Education \% education beyond primary school for draftees.
\item
  {[},5{]} Catholic \% `catholic' (as opposed to `protestant').
\item
  {[},6{]} Infant.Mortality live births who live less than 1 year.
\end{itemize}

All variables but `Fertility' give proportions of the population.

\hypertarget{calling-lm}{%
\subsection{\texorpdfstring{Calling \texttt{lm}}{Calling lm}}\label{calling-lm}}

\texttt{summary(lm(Fertility\ \textasciitilde{}\ .\ ,\ data\ =\ swiss))}

\begin{verbatim}
##                    Estimate  Std. Error
## (Intercept)      66.9151817 10.70603759
## Agriculture      -0.1721140  0.07030392
## Examination      -0.2580082  0.25387820
## Education        -0.8709401  0.18302860
## Catholic          0.1041153  0.03525785
## Infant.Mortality  1.0770481  0.38171965
\end{verbatim}

\hypertarget{example-interpretation}{%
\subsection{Example interpretation}\label{example-interpretation}}

\begin{itemize}
\tightlist
\item
  Agriculture is expressed in percentages (0 - 100)
\item
  Estimate is -0.1721.
\item
  We estimate an expected 0.17 decrease in standardized fertility for every 1\% increase in percentage of males involved in agriculture in holding the remaining variables constant.
\item
  The t-test for \(H_0: \beta_{Agri} = 0\) versus \(H_a: \beta_{Agri} \neq 0\) is significant.
\item
  Interestingly, the unadjusted estimate is
\end{itemize}

\begin{Shaded}
\begin{Highlighting}[]
\KeywordTok{summary}\NormalTok{(}\KeywordTok{lm}\NormalTok{(Fertility }\OperatorTok{~}\StringTok{ }\NormalTok{Agriculture, }\DataTypeTok{data =}\NormalTok{ swiss))}\OperatorTok{$}\NormalTok{coefficients}
\end{Highlighting}
\end{Shaded}

\begin{verbatim}
##               Estimate Std. Error   t value     Pr(>|t|)
## (Intercept) 60.3043752 4.25125562 14.185074 3.216304e-18
## Agriculture  0.1942017 0.07671176  2.531577 1.491720e-02
\end{verbatim}

How can adjustment reverse the sign of an effect? Let's try a simulation.

\begin{Shaded}
\begin{Highlighting}[]
\NormalTok{n <-}\StringTok{ }\DecValTok{100}\NormalTok{; x2 <-}\StringTok{ }\DecValTok{1} \OperatorTok{:}\StringTok{ }\NormalTok{n; x1 <-}\StringTok{ }\FloatTok{.01} \OperatorTok{*}\StringTok{ }\NormalTok{x2 }\OperatorTok{+}\StringTok{ }\KeywordTok{runif}\NormalTok{(n, }\FloatTok{-.1}\NormalTok{, }\FloatTok{.1}\NormalTok{); }
\NormalTok{y =}\StringTok{ }\OperatorTok{-}\NormalTok{x1 }\OperatorTok{+}\StringTok{ }\NormalTok{x2 }\OperatorTok{+}\StringTok{ }\KeywordTok{rnorm}\NormalTok{(n, }\DataTypeTok{sd =} \FloatTok{.01}\NormalTok{)}
\KeywordTok{summary}\NormalTok{(}\KeywordTok{lm}\NormalTok{(y }\OperatorTok{~}\StringTok{ }\NormalTok{x1))}\OperatorTok{$}\NormalTok{coef}
\end{Highlighting}
\end{Shaded}

\begin{verbatim}
##              Estimate Std. Error   t value     Pr(>|t|)
## (Intercept)  2.023307   1.183434  1.709691 9.048763e-02
## x1          94.832083   2.021226 46.918109 5.838380e-69
\end{verbatim}

\begin{Shaded}
\begin{Highlighting}[]
\KeywordTok{summary}\NormalTok{(}\KeywordTok{lm}\NormalTok{(y }\OperatorTok{~}\StringTok{ }\NormalTok{x1 }\OperatorTok{+}\StringTok{ }\NormalTok{x2))}\OperatorTok{$}\NormalTok{coef}
\end{Highlighting}
\end{Shaded}

\begin{verbatim}
##                 Estimate   Std. Error      t value      Pr(>|t|)
## (Intercept) -0.001610232 0.0019330674   -0.8329931  4.068950e-01
## x1          -1.001865718 0.0159178036  -62.9399471  1.843350e-80
## x2           1.000027376 0.0001625961 6150.3778670 5.525048e-273
\end{verbatim}

\begin{center}\includegraphics{thesis_files/figure-latex/unnamed-chunk-11-1} \end{center}

\hypertarget{back-to-this-data-set}{%
\subsection{Back to this data set}\label{back-to-this-data-set}}

\begin{itemize}
\tightlist
\item
  The sign reverses itself with the inclusion of Examination and Education, but of which are negatively correlated with Agriculture.
\item
  The percent of males in the province working in agriculture is negatively related to educational attainment (correlation of -0.6395225) and Education and Examination (correlation of 0.6984153) are obviously measuring similar things.

  \begin{itemize}
  \tightlist
  \item
    Is the positive marginal an artifact for not having accounted for, say, Education level? (Education does have a stronger effect, by the way.)
  \end{itemize}
\item
  At the minimum, anyone claiming that provinces that are more agricultural have higher fertility rates would immediately be open to criticism.
\end{itemize}

\hypertarget{what-if-we-include-an-unnecessary-variable}{%
\subsection{What if we include an unnecessary variable?}\label{what-if-we-include-an-unnecessary-variable}}

z adds no new linear information, since it's a linear
combination of variables already included. R just drops
terms that are linear combinations of other terms.

\begin{Shaded}
\begin{Highlighting}[]
\NormalTok{z <-}\StringTok{ }\NormalTok{swiss}\OperatorTok{$}\NormalTok{Agriculture }\OperatorTok{+}\StringTok{ }\NormalTok{swiss}\OperatorTok{$}\NormalTok{Education}
\KeywordTok{lm}\NormalTok{(Fertility }\OperatorTok{~}\StringTok{ }\NormalTok{. }\OperatorTok{+}\StringTok{ }\NormalTok{z, }\DataTypeTok{data =}\NormalTok{ swiss)[[}\DecValTok{1}\NormalTok{]][}\DecValTok{1}\OperatorTok{:}\DecValTok{3}\NormalTok{]}
\end{Highlighting}
\end{Shaded}

\begin{verbatim}
## (Intercept) Agriculture Examination 
##  66.9151817  -0.1721140  -0.2580082
\end{verbatim}

\hypertarget{dummy-variables-are-smart}{%
\subsection{Dummy variables are smart}\label{dummy-variables-are-smart}}

\begin{itemize}
\tightlist
\item
  Consider the linear model
  \[
  Y_i = \beta_0 + X_{i1} \beta_1 + \epsilon_{i}
  \]
  where each \(X_{i1}\) is binary so that it is a 1 if measurement \(i\) is in a group and 0 otherwise. (Treated versus not in a clinical trial, for example.)
\item
  Then for people in the group \(E[Y_i] = \beta_0 + \beta_1\)
\item
  And for people not in the group \(E[Y_i] = \beta_0\)
\item
  The LS fits work out to be \(\hat \beta_0 + \hat \beta_1\) is the mean for those in the group and \(\hat \beta_0\) is the mean for those not in the group.
\item
  \(\beta_1\) is interpretted as the increase or decrease in the mean comparing those in the group to those not.
\item
  Note including a binary variable that is 1 for those not in the group would be redundant. It would create three parameters to describe two means.
\end{itemize}

\hypertarget{more-than-2-levels}{%
\subsection{More than 2 levels}\label{more-than-2-levels}}

\begin{itemize}
\tightlist
\item
  Consider a multilevel factor level. For didactic reasons, let's say a three level factor (example, US political party affiliation: Republican, Democrat, Independent)
\item
  \(Y_i = \beta_0 + X_{i1} \beta_1 + X_{i2} \beta_2 + \epsilon_i\).
\item
  \(X_{i1}\) is 1 for Republicans and 0 otherwise.
\item
  \(X_{i2}\) is 1 for Democrats and 0 otherwise.
\item
  If \(i\) is Republican \(E[Y_i] = \beta_0 +\beta_1\)
\item
  If \(i\) is Democrat \(E[Y_i] = \beta_0 + \beta_2\).
\item
  If \(i\) is Independent \(E[Y_i] = \beta_0\).
\item
  \(\beta_1\) compares Republicans to Independents.
\item
  \(\beta_2\) compares Democrats to Independents.
\item
  \(\beta_1 - \beta_2\) compares Republicans to Democrats.
\item
  (Choice of reference category changes the interpretation.)
\end{itemize}

\hypertarget{insect-sprays}{%
\subsection{Insect Sprays}\label{insect-sprays}}

\begin{center}\includegraphics{thesis_files/figure-latex/unnamed-chunk-13-1} \end{center}

\hypertarget{linear-model-fit-group-a-is-the-reference}{%
\subsection{Linear model fit, group A is the reference}\label{linear-model-fit-group-a-is-the-reference}}

\begin{Shaded}
\begin{Highlighting}[]
\KeywordTok{summary}\NormalTok{(}\KeywordTok{lm}\NormalTok{(count }\OperatorTok{~}\StringTok{ }\NormalTok{spray, }\DataTypeTok{data =}\NormalTok{ InsectSprays))}\OperatorTok{$}\NormalTok{coef}
\end{Highlighting}
\end{Shaded}

\begin{verbatim}
##                Estimate Std. Error    t value     Pr(>|t|)
## (Intercept)  14.5000000   1.132156 12.8074279 1.470512e-19
## sprayB        0.8333333   1.601110  0.5204724 6.044761e-01
## sprayC      -12.4166667   1.601110 -7.7550382 7.266893e-11
## sprayD       -9.5833333   1.601110 -5.9854322 9.816910e-08
## sprayE      -11.0000000   1.601110 -6.8702352 2.753922e-09
## sprayF        2.1666667   1.601110  1.3532281 1.805998e-01
\end{verbatim}

\hypertarget{hard-coding-the-dummy-variables}{%
\subsection{Hard coding the dummy variables}\label{hard-coding-the-dummy-variables}}

\begin{Shaded}
\begin{Highlighting}[]
\KeywordTok{summary}\NormalTok{(}\KeywordTok{lm}\NormalTok{(count }\OperatorTok{~}\StringTok{ }
\StringTok{             }\KeywordTok{I}\NormalTok{(}\DecValTok{1} \OperatorTok{*}\StringTok{ }\NormalTok{(spray }\OperatorTok{==}\StringTok{ 'B'}\NormalTok{)) }\OperatorTok{+}\StringTok{ }\KeywordTok{I}\NormalTok{(}\DecValTok{1} \OperatorTok{*}\StringTok{ }\NormalTok{(spray }\OperatorTok{==}\StringTok{ 'C'}\NormalTok{)) }\OperatorTok{+}\StringTok{ }
\StringTok{             }\KeywordTok{I}\NormalTok{(}\DecValTok{1} \OperatorTok{*}\StringTok{ }\NormalTok{(spray }\OperatorTok{==}\StringTok{ 'D'}\NormalTok{)) }\OperatorTok{+}\StringTok{ }\KeywordTok{I}\NormalTok{(}\DecValTok{1} \OperatorTok{*}\StringTok{ }\NormalTok{(spray }\OperatorTok{==}\StringTok{ 'E'}\NormalTok{)) }\OperatorTok{+}
\StringTok{             }\KeywordTok{I}\NormalTok{(}\DecValTok{1} \OperatorTok{*}\StringTok{ }\NormalTok{(spray }\OperatorTok{==}\StringTok{ 'F'}\NormalTok{))}
\NormalTok{           , }\DataTypeTok{data =}\NormalTok{ InsectSprays))}\OperatorTok{$}\NormalTok{coef}
\end{Highlighting}
\end{Shaded}

\begin{verbatim}
##                          Estimate Std. Error    t value     Pr(>|t|)
## (Intercept)            14.5000000   1.132156 12.8074279 1.470512e-19
## I(1 * (spray == "B"))   0.8333333   1.601110  0.5204724 6.044761e-01
## I(1 * (spray == "C")) -12.4166667   1.601110 -7.7550382 7.266893e-11
## I(1 * (spray == "D"))  -9.5833333   1.601110 -5.9854322 9.816910e-08
## I(1 * (spray == "E")) -11.0000000   1.601110 -6.8702352 2.753922e-09
## I(1 * (spray == "F"))   2.1666667   1.601110  1.3532281 1.805998e-01
\end{verbatim}

\hypertarget{what-if-we-include-all-6}{%
\subsection{What if we include all 6?}\label{what-if-we-include-all-6}}

\begin{Shaded}
\begin{Highlighting}[]
\KeywordTok{lm}\NormalTok{(count }\OperatorTok{~}\StringTok{ }
\StringTok{   }\KeywordTok{I}\NormalTok{(}\DecValTok{1} \OperatorTok{*}\StringTok{ }\NormalTok{(spray }\OperatorTok{==}\StringTok{ 'B'}\NormalTok{)) }\OperatorTok{+}\StringTok{ }\KeywordTok{I}\NormalTok{(}\DecValTok{1} \OperatorTok{*}\StringTok{ }\NormalTok{(spray }\OperatorTok{==}\StringTok{ 'C'}\NormalTok{)) }\OperatorTok{+}\StringTok{  }
\StringTok{   }\KeywordTok{I}\NormalTok{(}\DecValTok{1} \OperatorTok{*}\StringTok{ }\NormalTok{(spray }\OperatorTok{==}\StringTok{ 'D'}\NormalTok{)) }\OperatorTok{+}\StringTok{ }\KeywordTok{I}\NormalTok{(}\DecValTok{1} \OperatorTok{*}\StringTok{ }\NormalTok{(spray }\OperatorTok{==}\StringTok{ 'E'}\NormalTok{)) }\OperatorTok{+}
\StringTok{   }\KeywordTok{I}\NormalTok{(}\DecValTok{1} \OperatorTok{*}\StringTok{ }\NormalTok{(spray }\OperatorTok{==}\StringTok{ 'F'}\NormalTok{)) }\OperatorTok{+}\StringTok{ }\KeywordTok{I}\NormalTok{(}\DecValTok{1} \OperatorTok{*}\StringTok{ }\NormalTok{(spray }\OperatorTok{==}\StringTok{ 'A'}\NormalTok{)), }\DataTypeTok{data =}\NormalTok{ InsectSprays)}
\end{Highlighting}
\end{Shaded}

\hypertarget{what-if-we-omit-the-intercept}{%
\subsection{What if we omit the intercept?}\label{what-if-we-omit-the-intercept}}

\begin{Shaded}
\begin{Highlighting}[]
\KeywordTok{summary}\NormalTok{(}\KeywordTok{lm}\NormalTok{(count }\OperatorTok{~}\StringTok{ }\NormalTok{spray }\OperatorTok{-}\StringTok{ }\DecValTok{1}\NormalTok{, }\DataTypeTok{data =}\NormalTok{ InsectSprays))}\OperatorTok{$}\NormalTok{coef}
\end{Highlighting}
\end{Shaded}

\begin{verbatim}
##         Estimate Std. Error   t value     Pr(>|t|)
## sprayA 14.500000   1.132156 12.807428 1.470512e-19
## sprayB 15.333333   1.132156 13.543487 1.001994e-20
## sprayC  2.083333   1.132156  1.840148 7.024334e-02
## sprayD  4.916667   1.132156  4.342749 4.953047e-05
## sprayE  3.500000   1.132156  3.091448 2.916794e-03
## sprayF 16.666667   1.132156 14.721181 1.573471e-22
\end{verbatim}

\begin{Shaded}
\begin{Highlighting}[]
\KeywordTok{unique}\NormalTok{(}\KeywordTok{ave}\NormalTok{(InsectSprays}\OperatorTok{$}\NormalTok{count, InsectSprays}\OperatorTok{$}\NormalTok{spray))}
\end{Highlighting}
\end{Shaded}

\begin{verbatim}
## [1] 14.500000 15.333333  2.083333  4.916667  3.500000 16.666667
\end{verbatim}

\hypertarget{summary}{%
\subsection{Summary}\label{summary}}

\begin{itemize}
\tightlist
\item
  If we treat Spray as a factor, R includes an intercept and omits the alphabetically first level of the factor.

  \begin{itemize}
  \tightlist
  \item
    All t-tests are for comparisons of Sprays versus Spray A.
  \item
    Emprirical mean for A is the intercept.
  \item
    Other group means are the itc plus their coefficient.
  \end{itemize}
\item
  If we omit an intercept, then it includes terms for all levels of the factor.

  \begin{itemize}
  \tightlist
  \item
    Group means are the coefficients.
  \item
    Tests are tests of whether the groups are different than zero. (Are the expected counts zero for that spray.)
  \end{itemize}
\item
  If we want comparisons between, Spray B and C, say we could refit the model with C (or B) as the reference level.
\end{itemize}

Reordering the levels

\begin{Shaded}
\begin{Highlighting}[]
\NormalTok{spray2 <-}\StringTok{ }\KeywordTok{relevel}\NormalTok{(InsectSprays}\OperatorTok{$}\NormalTok{spray, }\StringTok{"C"}\NormalTok{)}
\KeywordTok{summary}\NormalTok{(}\KeywordTok{lm}\NormalTok{(count }\OperatorTok{~}\StringTok{ }\NormalTok{spray2, }\DataTypeTok{data =}\NormalTok{ InsectSprays))}\OperatorTok{$}\NormalTok{coef}
\end{Highlighting}
\end{Shaded}

\begin{verbatim}
##              Estimate Std. Error  t value     Pr(>|t|)
## (Intercept)  2.083333   1.132156 1.840148 7.024334e-02
## spray2A     12.416667   1.601110 7.755038 7.266893e-11
## spray2B     13.250000   1.601110 8.275511 8.509776e-12
## spray2D      2.833333   1.601110 1.769606 8.141205e-02
## spray2E      1.416667   1.601110 0.884803 3.794750e-01
## spray2F     14.583333   1.601110 9.108266 2.794343e-13
\end{verbatim}

Doing it manually
Equivalently
\[Var(\hat \beta_B - \hat \beta_C) = Var(\hat \beta_B) + Var(\hat \beta_C) - 2 Cov(\hat \beta_B, \hat \beta_C)\]

\begin{Shaded}
\begin{Highlighting}[]
\NormalTok{fit <-}\StringTok{ }\KeywordTok{lm}\NormalTok{(count }\OperatorTok{~}\StringTok{ }\NormalTok{spray, }\DataTypeTok{data =}\NormalTok{ InsectSprays) }\CommentTok{#A is ref}
\NormalTok{bbmbc <-}\StringTok{ }\KeywordTok{coef}\NormalTok{(fit)[}\DecValTok{2}\NormalTok{] }\OperatorTok{-}\StringTok{ }\KeywordTok{coef}\NormalTok{(fit)[}\DecValTok{3}\NormalTok{] }\CommentTok{#B - C}
\NormalTok{temp <-}\StringTok{ }\KeywordTok{summary}\NormalTok{(fit) }
\NormalTok{se <-}\StringTok{ }\NormalTok{temp}\OperatorTok{$}\NormalTok{sigma }\OperatorTok{*}\StringTok{ }
\StringTok{      }\KeywordTok{sqrt}\NormalTok{(temp}\OperatorTok{$}\NormalTok{cov.unscaled[}\DecValTok{2}\NormalTok{, }\DecValTok{2}\NormalTok{] }\OperatorTok{+}\StringTok{ }
\StringTok{      }\NormalTok{temp}\OperatorTok{$}\NormalTok{cov.unscaled[}\DecValTok{3}\NormalTok{,}\DecValTok{3}\NormalTok{] }\OperatorTok{-}\StringTok{ }
\StringTok{      }\DecValTok{2} \OperatorTok{*}\NormalTok{temp}\OperatorTok{$}\NormalTok{cov.unscaled[}\DecValTok{2}\NormalTok{,}\DecValTok{3}\NormalTok{])}
\NormalTok{t <-}\StringTok{ }\NormalTok{(bbmbc) }\OperatorTok{/}\StringTok{ }\NormalTok{se}
\NormalTok{p <-}\StringTok{ }\KeywordTok{pt}\NormalTok{(}\OperatorTok{-}\KeywordTok{abs}\NormalTok{(t), }\DataTypeTok{df =}\NormalTok{ fit}\OperatorTok{$}\NormalTok{df)}
\NormalTok{out <-}\StringTok{ }\KeywordTok{c}\NormalTok{(bbmbc, se, t, p)}
\KeywordTok{names}\NormalTok{(out) <-}\StringTok{ }\KeywordTok{c}\NormalTok{(}\StringTok{"B - C"}\NormalTok{, }\StringTok{"SE"}\NormalTok{, }\StringTok{"T"}\NormalTok{, }\StringTok{"P"}\NormalTok{)}
\KeywordTok{round}\NormalTok{(out, }\DecValTok{3}\NormalTok{)}
\end{Highlighting}
\end{Shaded}

\begin{verbatim}
##  B - C     SE      T      P 
## 13.250  1.601  8.276  0.000
\end{verbatim}

\hypertarget{other-thoughts-on-this-data}{%
\subsection{Other thoughts on this data}\label{other-thoughts-on-this-data}}

\begin{itemize}
\tightlist
\item
  Counts are bounded from below by 0, violates the assumption of normality of the errors.

  \begin{itemize}
  \tightlist
  \item
    Also there are counts near zero, so both the actual assumption and the intent of the assumption are violated.
  \end{itemize}
\item
  Variance does not appear to be constant.
\item
  Perhaps taking logs of the counts would help.

  \begin{itemize}
  \tightlist
  \item
    There are 0 counts, so maybe log(Count + 1)
  \end{itemize}
\item
  Also, we'll cover Poisson GLMs for fitting count data.
\end{itemize}

WHO childhood hunger data

\begin{Shaded}
\begin{Highlighting}[]
\CommentTok{#url <- "http://apps.who.int/gho/athena/data/GHO/WHOSIS_000008.csv?profile=text&filter=COUNTRY:*;SEX:*"}
\CommentTok{#download.file(url, "hunger.csv", method="curl")}
\NormalTok{hunger <-}\StringTok{ }\KeywordTok{read.csv}\NormalTok{(}\StringTok{'01-data/sheets/hunger.csv'}\NormalTok{)}
\NormalTok{hunger <-}\StringTok{ }\NormalTok{hunger[hunger}\OperatorTok{$}\NormalTok{Sex}\OperatorTok{!=}\StringTok{"Both sexes"}\NormalTok{,]}
\KeywordTok{kable}\NormalTok{(}\KeywordTok{head}\NormalTok{(hunger[,}\DecValTok{1}\OperatorTok{:}\DecValTok{2}\NormalTok{],}\DecValTok{3}\NormalTok{))}
\end{Highlighting}
\end{Shaded}

\begin{tabular}{l|l}
\hline
Indicator & Data.Source\\
\hline
Children aged <5 years underweight (\%) & NLIS\_310044\\
\hline
Children aged <5 years underweight (\%) & NLIS\_310233\\
\hline
Children aged <5 years underweight (\%) & NLIS\_312902\\
\hline
\end{tabular}

Plot percent hungry versus time

\begin{Shaded}
\begin{Highlighting}[]
\NormalTok{lm1 <-}\StringTok{ }\KeywordTok{lm}\NormalTok{(hunger}\OperatorTok{$}\NormalTok{Numeric }\OperatorTok{~}\StringTok{ }\NormalTok{hunger}\OperatorTok{$}\NormalTok{Year)}
\KeywordTok{plot}\NormalTok{(hunger}\OperatorTok{$}\NormalTok{Year,hunger}\OperatorTok{$}\NormalTok{Numeric,}\DataTypeTok{pch=}\DecValTok{19}\NormalTok{,}\DataTypeTok{col=}\StringTok{"blue"}\NormalTok{)}
\end{Highlighting}
\end{Shaded}

\begin{center}\includegraphics{thesis_files/figure-latex/unnamed-chunk-20-1} \end{center}

Linear model

\[Hu_i = b_0 + b_1 Y_i + e_i\]

\(b_0\) = percent hungry at Year 0

\(b_1\) = decrease in percent hungry per year

\(e_i\) = everything we didn't measure

Add the linear model

\begin{Shaded}
\begin{Highlighting}[]
\NormalTok{lm1 <-}\StringTok{ }\KeywordTok{lm}\NormalTok{(hunger}\OperatorTok{$}\NormalTok{Numeric }\OperatorTok{~}\StringTok{ }\NormalTok{hunger}\OperatorTok{$}\NormalTok{Year)}
\KeywordTok{plot}\NormalTok{(hunger}\OperatorTok{$}\NormalTok{Year,hunger}\OperatorTok{$}\NormalTok{Numeric,}\DataTypeTok{pch=}\DecValTok{19}\NormalTok{,}\DataTypeTok{col=}\StringTok{"blue"}\NormalTok{)}
\KeywordTok{lines}\NormalTok{(hunger}\OperatorTok{$}\NormalTok{Year,lm1}\OperatorTok{$}\NormalTok{fitted,}\DataTypeTok{lwd=}\DecValTok{3}\NormalTok{,}\DataTypeTok{col=}\StringTok{"darkgrey"}\NormalTok{)}
\end{Highlighting}
\end{Shaded}

\includegraphics{thesis_files/figure-latex/unnamed-chunk-21-1.pdf}

\begin{Shaded}
\begin{Highlighting}[]
\KeywordTok{plot}\NormalTok{(hunger}\OperatorTok{$}\NormalTok{Year,hunger}\OperatorTok{$}\NormalTok{Numeric,}\DataTypeTok{pch=}\DecValTok{19}\NormalTok{)}
\end{Highlighting}
\end{Shaded}

\begin{center}\includegraphics{thesis_files/figure-latex/unnamed-chunk-22-1} \end{center}

Now two lines

\[HuF_i = bf_0 + bf_1 YF_i + ef_i\]

\(bf_0\) = percent of girls hungry at Year 0

\(bf_1\) = decrease in percent of girls hungry per year

\(ef_i\) = everything we didn't measure

\[HuM_i = bm_0 + bm_1 YM_i + em_i\]

\(bm_0\) = percent of boys hungry at Year 0

\(bm_1\) = decrease in percent of boys hungry per year

\(em_i\) = everything we didn't measure

\begin{Shaded}
\begin{Highlighting}[]
\NormalTok{lmM <-}\StringTok{ }\KeywordTok{lm}\NormalTok{(hunger}\OperatorTok{$}\NormalTok{Numeric[hunger}\OperatorTok{$}\NormalTok{Sex}\OperatorTok{==}\StringTok{"Male"}\NormalTok{] }\OperatorTok{~}\StringTok{ }
\StringTok{            }\NormalTok{hunger}\OperatorTok{$}\NormalTok{Year[hunger}\OperatorTok{$}\NormalTok{Sex}\OperatorTok{==}\StringTok{"Male"}\NormalTok{])}
\NormalTok{lmF <-}\StringTok{ }\KeywordTok{lm}\NormalTok{(hunger}\OperatorTok{$}\NormalTok{Numeric[hunger}\OperatorTok{$}\NormalTok{Sex}\OperatorTok{==}\StringTok{"Female"}\NormalTok{] }\OperatorTok{~}
\StringTok{            }\NormalTok{hunger}\OperatorTok{$}\NormalTok{Year[hunger}\OperatorTok{$}\NormalTok{Sex}\OperatorTok{==}\StringTok{"Female"}\NormalTok{])}
\KeywordTok{plot}\NormalTok{(hunger}\OperatorTok{$}\NormalTok{Year,hunger}\OperatorTok{$}\NormalTok{Numeric,}\DataTypeTok{pch=}\DecValTok{19}\NormalTok{)}
\end{Highlighting}
\end{Shaded}

\begin{center}\includegraphics{thesis_files/figure-latex/unnamed-chunk-23-1} \end{center}

Two lines, same slope

\[Hu_i = b_0 + b_1 \mathbb{1}(Sex_i="Male") + b_2 Y_i + e^*_i\]

\(b_0\) - percent hungry at year zero for females

\(b_0 + b_1\) - percent hungry at year zero for males

\(b_2\) - change in percent hungry (for either males or females) in one year

\(e^*_i\) - everything we didn't measure

Two lines, same slope in R

\begin{Shaded}
\begin{Highlighting}[]
\NormalTok{lmBoth <-}\StringTok{ }\KeywordTok{lm}\NormalTok{(hunger}\OperatorTok{$}\NormalTok{Numeric }\OperatorTok{~}\StringTok{ }\NormalTok{hunger}\OperatorTok{$}\NormalTok{Year }\OperatorTok{+}\StringTok{ }\NormalTok{hunger}\OperatorTok{$}\NormalTok{Sex)}
\KeywordTok{plot}\NormalTok{(hunger}\OperatorTok{$}\NormalTok{Year,hunger}\OperatorTok{$}\NormalTok{Numeric,}\DataTypeTok{pch=}\DecValTok{19}\NormalTok{)}
\end{Highlighting}
\end{Shaded}

\begin{center}\includegraphics{thesis_files/figure-latex/unnamed-chunk-24-1} \end{center}

Two lines, different slopes (interactions)

\[Hu_i = b_0 + b_1 \mathbb{1}(Sex_i="Male") + b_2 Y_i + b_3 \mathbb{1}(Sex_i="Male")\times Y_i + e^+_i\]

\(b_0\) - percent hungry at year zero for females

\(b_0 + b_1\) - percent hungry at year zero for males

\(b_2\) - change in percent hungry (females) in one year

\(b_2 + b_3\) - change in percent hungry (males) in one year

\(e^+_i\) - everything we didn't measure

Two lines, different slopes in R

\begin{Shaded}
\begin{Highlighting}[]
\NormalTok{lmBoth <-}\StringTok{ }\KeywordTok{lm}\NormalTok{(hunger}\OperatorTok{$}\NormalTok{Numeric }\OperatorTok{~}\StringTok{ }\NormalTok{hunger}\OperatorTok{$}\NormalTok{Year }\OperatorTok{+}\StringTok{ }\NormalTok{hunger}\OperatorTok{$}\NormalTok{Sex }\OperatorTok{+}\StringTok{ }
\StringTok{               }\NormalTok{hunger}\OperatorTok{$}\NormalTok{Sex}\OperatorTok{*}\NormalTok{hunger}\OperatorTok{$}\NormalTok{Year)}
\KeywordTok{plot}\NormalTok{(hunger}\OperatorTok{$}\NormalTok{Year,hunger}\OperatorTok{$}\NormalTok{Numeric,}\DataTypeTok{pch=}\DecValTok{19}\NormalTok{)}
\end{Highlighting}
\end{Shaded}

\begin{center}\includegraphics{thesis_files/figure-latex/lmBothChunk-1} \end{center}

Two lines, different slopes in R

\begin{Shaded}
\begin{Highlighting}[]
\KeywordTok{summary}\NormalTok{(lmBoth)[}\DecValTok{4}\NormalTok{]}
\end{Highlighting}
\end{Shaded}

\begin{verbatim}
## $coefficients
##                                Estimate   Std. Error    t value     Pr(>|t|)
## (Intercept)                603.50579986 171.05519432  3.5281349 0.0004386682
## hunger$Year                 -0.29339638   0.08546675 -3.4328718 0.0006231690
## hunger$SexMale              61.94771998 241.90857572  0.2560791 0.7979455842
## hunger$Year:hunger$SexMale  -0.03000132   0.12086823 -0.2482151 0.8040219874
\end{verbatim}

Interpretting a continuous interaction
\[
E[Y_i | X_{1i}=x_1, X_{2i}=x_2] = \beta_0 + \beta_1 x_{1} + \beta_2 x_{2} + \beta_3 x_{1}x_{2}
\]
Holding \(X_2\) constant we have
\[
E[Y_i | X_{1i}=x_1+1, X_{2i}=x_2]-E[Y_i | X_{1i}=x_1, X_{2i}=x_2]
= \beta_1 + \beta_3 x_{2} 
\]
And thus the expected change in \(Y\) per unit change in \(X_1\) holding all else constant is not constant. \(\beta_1\) is the slope when \(x_{2} = 0\). Note further that:
\[
E[Y_i | X_{1i}=x_1+1, X_{2i}=x_2+1]-E[Y_i | X_{1i}=x_1, X_{2i}=x_2+1]
\]
\[
-E[Y_i | X_{1i}=x_1+1, X_{2i}=x_2]-E[Y_i | X_{1i}=x_1, X_{2i}=x_2]
\]
\[
=\beta_3  
\]
Thus, \(\beta_3\) is the change in the expected change in \(Y\) per unit change in \(X_1\), per unit change in \(X_2\).

Or, the change in the slope relating \(X_1\) and \(Y\) per unit change in \(X_2\).

Example

\[Hu_i = b_0 + b_1 In_i + b_2 Y_i + b_3 In_i \times Y_i + e^+_i\]

\(b_0\) - percent hungry at year zero for children with whose parents have no income

\(b_1\) - change in percent hungry for each dollar of income in year zero

\(b_2\) - change in percent hungry in one year for children whose parents have no income

\(b_3\) - increased change in percent hungry by year for each dollar of income - e.g.~if income is \$10,000, then change in percent hungry in one year will be

\[b_2 + 1e4 \times b_3\]

\(e^+_i\) - everything we didn't measure

\phantompart

\chapter*[Conclusão]{CONSIDERAÇÕES FINAIS}
\addcontentsline{toc}{chapter}{CONSIDERAÇÕES FINAIS}

\lipsum[31-33]

\postextual

\addtocontents{toc}{\vspace{-2pt}}

\printbibliography

\postextual

\addtocontents{toc}{\vspace{-2pt}}

\ifthenelse{\equal{\terApendice}{Sim}}
{\begin{apendicesenv}

\renewcommand{\thechapter}{\arabic{chapter}}

\chapter{ESCOLHA DO MATERIAL DE IMPRESSÃO}

\lipsum[30]

\end{apendicesenv}
}{}

\ifthenelse{\equal{\terAnexo}{Sim}}{
\begin{anexosenv}

\renewcommand{\thechapter}{\arabic{chapter}}
        
\chapter{TABELAS DE VALORES}
\lipsum[31] 

\chapter{GRÁFICOS DE BALANCEAMENTO}

\lipsum[32] 

\end{anexosenv}
}{}

\ifthenelse{\equal{\terIndiceR}{Sim}}{
\phantompart
\printindex
}{}

\printbibliography

\end{document}
